\documentclass[12pt]{article}
\usepackage{amsmath, amssymb}
\usepackage{geometry}
\geometry{margin=1in}

\title{Justification de la normalisation des écarts premiers \\
	\textit{Justification of the normalization of prime gaps}}
\author{Michel Monfette mycmon@gmail.com}
\date{2026}

\begin{document}
	
	\maketitle
	
	% ============================================================
	% ======================= VERSION FR ==========================
	% ============================================================
	
	\section*{Version française}
	
	\section{Justification de la normalisation}
	
	Dans cette section, nous justifions la définition
	
	
	\[
	g_n = \frac{p_{n+1} - p_n}{\log(p_n)},
	\]
	
	
	où $p_n$ désigne le $n$-ième nombre premier. Cette normalisation joue un rôle central dans la construction d'une grammaire stationnaire des écarts premiers.
	
	\subsection{1. Origine heuristique : théorème des nombres premiers}
	
	Le point de départ est le théorème des nombres premiers, qui affirme que
	
	
	\[
	\pi(x) \sim \frac{x}{\log x},
	\]
	
	
	où $\pi(x)$ est le nombre de nombres premiers inférieurs ou égaux à $x$. Cette relation implique que, autour d'un grand réel $x$, la \emph{densité} des nombres premiers est approximativement
	
	
	\[
	\text{densité} \approx \frac{1}{\log x}.
	\]
	
	
	L'écart moyen entre deux nombres premiers voisins dans un voisinage de $x$ est donc de l'ordre de
	
	
	\[
	\text{écart moyen} \approx \log x.
	\]
	
	
	
	En appliquant cette heuristique à la suite des nombres premiers $(p_n)$, on obtient
	
	
	\[
	p_{n+1} - p_n \approx \log(p_n),
	\]
	
	
	ce qui suggère que la taille typique de l'écart entre $p_n$ et $p_{n+1}$ est de l'ordre de $\log(p_n)$.
	
	\subsection{2. Normalisation sans dimension}
	
	La quantité brute $p_{n+1} - p_n$ croît en moyenne avec $p_n$. Pour comparer des écarts situés à des échelles numériques très différentes (par exemple autour de $10^4$, $10^6$ ou $10^8$), il est naturel de les rapporter à leur taille moyenne attendue.
	
	On définit donc
	
	
	\[
	g_n = \frac{p_{n+1} - p_n}{\log(p_n)}.
	\]
	
	
	Si l'approximation heuristique
	
	
	\[
	p_{n+1} - p_n \approx \log(p_n)
	\]
	
	
	est valide en moyenne, alors on s'attend à ce que
	
	
	\[
	g_n \approx 1
	\]
	
	
	pour une large proportion d'indices $n$. La quantité $g_n$ est ainsi une version \emph{normalisée} de l'écart, sans dimension, qui permet de distinguer :
	\begin{itemize}
		\item les écarts plus petits que la moyenne ($g_n \ll 1$),
		\item les écarts typiques ($g_n \approx 1$),
		\item les écarts plus grands que la moyenne ($g_n \gg 1$).
	\end{itemize}
	Cette normalisation rend les écarts comparables à toutes les échelles et prépare le terrain pour une discrétisation en symboles.
	
	\subsection{3. Validation empirique et stationnarité}
	
	L'analyse expérimentale montre que la distribution des valeurs de $g_n$ est remarquablement stable lorsque l'on considère des intervalles de plus en plus grands, par exemple :
	
	
	\[
	[10\,000, 20\,000], \quad [1\,000\,000, 2\,000\,000], \quad [10\,000\,000, 20\,000\,000].
	\]
	
	
	Les histogrammes de $g_n$ sur ces intervalles présentent des formes similaires, et les proportions des classes définies par la discrétisation de $g_n$ (symboles $a,b,c,d,\dots$) restent quasi constantes.
	
	Cette \emph{stationnarité empirique} justifie a posteriori le choix de la normalisation :
	
	
	\[
	g_n = \frac{p_{n+1} - p_n}{\log(p_n)}.
	\]
	
	
	Elle permet de construire une grammaire symbolique dont les règles de transition ne dépendent plus de l'échelle numérique, mais uniquement de la dynamique interne des écarts normalisés.
	
	% ============================================================
	% ======================= VERSION EN ==========================
	% ============================================================
	
	\newpage
	\section*{English Version}
	
	\section{Justification of the normalization}
	
	In this section, we justify the definition
	
	
	\[
	g_n = \frac{p_{n+1} - p_n}{\log(p_n)},
	\]
	
	
	where $p_n$ denotes the $n$-th prime number. This normalization plays a central role in the construction of a stationary grammar of prime gaps.
	
	\subsection{1. Heuristic origin: prime number theorem}
	
	The starting point is the prime number theorem, which states that
	
	
	\[
	\pi(x) \sim \frac{x}{\log x},
	\]
	
	
	where $\pi(x)$ is the number of primes less than or equal to $x$. This implies that, around a large real number $x$, the \emph{density} of primes is approximately
	
	
	\[
	\text{density} \approx \frac{1}{\log x}.
	\]
	
	
	The average gap between consecutive primes near $x$ is therefore of order
	
	
	\[
	\text{average gap} \approx \log x.
	\]
	
	
	
	Applying this heuristic to the sequence of primes $(p_n)$ yields
	
	
	\[
	p_{n+1} - p_n \approx \log(p_n),
	\]
	
	
	which suggests that the typical size of the gap between $p_n$ and $p_{n+1}$ is of order $\log(p_n)$.
	
	\subsection{2. Dimensionless normalization}
	
	The raw quantity $p_{n+1} - p_n$ grows on average with $p_n$. To compare gaps at very different numerical scales (for example around $10^4$, $10^6$, or $10^8$), it is natural to rescale them by their expected average size.
	
	We therefore define
	
	
	\[
	g_n = \frac{p_{n+1} - p_n}{\log(p_n)}.
	\]
	
	
	If the heuristic approximation
	
	
	\[
	p_{n+1} - p_n \approx \log(p_n)
	\]
	
	
	holds on average, then we expect
	
	
	\[
	g_n \approx 1
	\]
	
	
	for a large proportion of indices $n$. The quantity $g_n$ is thus a \emph{normalized}, dimensionless version of the gap, which allows us to distinguish:
	\begin{itemize}
		\item gaps smaller than the average ($g_n \ll 1$),
		\item typical gaps ($g_n \approx 1$),
		\item gaps larger than the average ($g_n \gg 1$).
	\end{itemize}
	This normalization makes gaps comparable across all scales and prepares the ground for discretization into symbols.
	
	\subsection{3. Empirical validation and stationarity}
	
	Experimental analysis shows that the distribution of the values $g_n$ is remarkably stable when considering larger and larger intervals, for example:
	
	
	\[
	[10\,000, 20\,000], \quad [1\,000\,000, 2\,000\,000], \quad [10\,000\,000, 20\,000\,000].
	\]
	
	
	Histograms of $g_n$ over these intervals exhibit similar shapes, and the proportions of the classes defined by the discretization of $g_n$ (symbols $a,b,c,d,\dots$) remain almost constant.
	
	This \emph{empirical stationarity} provides an a posteriori justification for the choice of the normalization
	
	
	\[
	g_n = \frac{p_{n+1} - p_n}{\log(p_n)}.
	\]
	
	
	It allows the construction of a symbolic grammar whose transition rules no longer depend on the numerical scale, but only on the internal dynamics of the normalized gaps.
	
\end{document}
