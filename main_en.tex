\documentclass[12pt]{article}

\usepackage[T1]{fontenc}
\usepackage[utf8]{inputenc}
\usepackage{lmodern}
\usepackage{amsmath, amssymb}
\usepackage{geometry}
\usepackage{setspace}
\usepackage{booktabs}

\geometry{margin=1in}
\setstretch{1.2}

\title{
	A Symbolic Grammar for Normalized Prime Gaps:\\
	An Empirical Markov Model
}

\author{Michel Monfette mycmon@gmail.com (Quebec, Canada)}
\date{January 2026}

\begin{document}
	
	\maketitle
	
	\begin{abstract}
		This short note presents an empirical structure observed in normalized prime gaps.
		After normalizing gaps by $\log(p)$, the resulting values cluster into stable ranges that can be discretized into a small symbolic alphabet.
		The frequencies of bigrams and trigrams in this symbolic sequence remain stable across large intervals.
		A simple first-order Markov model is constructed from these motifs.
		Combined with elementary arithmetic filters (mod 30 residues and removal of small prime multiples), the model generates simulated values that match actual prime gaps significantly better than random baselines.
		This work does not propose a theory of primes; it simply documents an empirical structure that may be of independent interest.
	\end{abstract}
	
	\section{Introduction}
	
	Prime gaps have been studied extensively from analytic and probabilistic perspectives.
	This work explores them from a different angle: symbolic discretization.
	The idea originated from a geometric intuition: primes greater than 5 occupy only eight residue classes modulo 30.
	This suggested that normalized prime gaps might exhibit stable symbolic patterns.
	
	The goal of this note is to describe this empirical structure and share it with the mathematical community.
	
	\section{Normalization of Prime Gaps}
	
	For consecutive primes $p_n$ and $p_{n+1}$, define the gap:
	
	
	\[
	\Delta_n = p_{n+1} - p_n.
	\]
	
	
	
	Normalize it using:
	
	
	\[
	g_n = \frac{\Delta_n}{\log(p_n)}.
	\]
	
	
	
	This normalization makes gaps comparable across large intervals.
	The values $g_n$ tend to cluster into a small number of ranges.
	
	\section{Symbolic Grammar}
	
	The normalized values are discretized into four symbols:
	
	\begin{itemize}
		\item a: very small gap
		\item b: small gap
		\item c: typical gap
		\item d: large gap
	\end{itemize}
	
	Across multiple intervals (1--2000, 20k--50k, 1M--2M), the most frequent motifs are:
	
	\subsection*{Bigrams}
	
	
	\[
	bb,\ ab,\ ba,\ bc,\ cb,\ aa,\ ca
	\]
	
	
	
	\subsection*{Trigrams}
	
	
	\[
	bbb,\ bba,\ bab,\ abb,\ bcb,\ bbc
	\]
	
	
	
	\subsection*{Tetragrams}
	
	
	\[
	bbbb,\ bbab,\ bcbb,\ bbba,\ cbba,\ abbb
	\]
	
	
	
	These motifs remain stable across scales, suggesting a symbolic structure.
	
	\section{Markov Model}
	
	A first-order Markov chain is built from bigram frequencies:
	
	
	\[
	P(s_{n+1} \mid s_n).
	\]
	
	
	
	Each symbol corresponds to a typical normalized gap value.
	Simulated sequences of symbols can be converted back into numerical gaps.
	
	\section{Arithmetic Filters}
	
	Two simple filters improve realism:
	
	\subsection*{Allowed residues modulo 30}
	Only the residues
	
	
	\[
	1,\ 7,\ 11,\ 13,\ 17,\ 19,\ 23,\ 29
	\]
	
	
	are allowed.
	
	\subsection*{Small-prime filter}
	Numbers divisible by
	
	
	\[
	7,\ 11,\ 13,\ 17,\ 19,\ 23,\ 29
	\]
	
	
	are rejected.
	
	These filters remove obvious composites.
	
	\section{Numerical Observations}
	
	In the interval $1$--$2$ million, with 100 simulated values:
	
	\begin{itemize}
		\item 30--41\% of simulated values are actual primes
		\item true density: 7.24\%
		\item performance: approximately $4.1 \times$ random
	\end{itemize}
	
	\section{Comparison with Granville}
	
	Granville's model predicts:
	
	
	\[
	P(n) \approx \frac{1.2}{\ln(n)} \approx 7\%.
	\]
	
	
	
	\begin{center}
		\begin{tabular}{l c}
			\toprule
			Model & Result \\
			\midrule
			Symbolic grammar & 30--41\% \\
			Granville & $\approx 7\%$ \\
			Random baseline & $\approx 7\%$ \\
			\bottomrule
		\end{tabular}
	\end{center}
	
	The symbolic model captures local motifs that Granville's global model does not attempt to describe.
	
	\section{Conclusion}
	
	This work does not propose a theory of prime numbers.
	It documents an empirical symbolic structure that appears stable across large intervals.
	The goal is simply to share this observation with the mathematical community so that the idea is not lost and may inspire further exploration.
	
\end{document}
