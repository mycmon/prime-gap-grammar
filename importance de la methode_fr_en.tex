\documentclass[12pt]{article}
\usepackage{amsmath, amssymb}
\usepackage{graphicx}
\usepackage{hyperref}
\usepackage{geometry}
\geometry{margin=1in}

\title{%
	Grammaire géométrique des écarts premiers \\
	\large Analyse symbolique, structure markovienne et dynamique modulo 30 \\
	\vspace{0.5cm}
	\textit{Geometric Grammar of Prime Gaps} \\
	\large Symbolic Analysis, Markov Structure, and Modulo 30 Dynamics
}
\author{Michel Monfette mycmon@gmail.com}
\date{2026}

\begin{document}
	
	\maketitle
	
	% --------------------------------------------------------------------
	\begin{abstract}
		\textbf{Résumé (FR).}  
		Cette étude propose une approche géométrique et symbolique des écarts entre nombres premiers. 
		En normalisant les écarts par la fonction logarithme, en les représentant dans un espace angulaire modulo~30 
		et en les discrétisant en un alphabet fini, il devient possible de construire une grammaire markovienne 
		révélant des régularités profondes et stables à travers plusieurs échelles numériques.
		
		\medskip
		\textbf{Abstract (EN).}  
		This study introduces a geometric and symbolic approach to the analysis of gaps between consecutive prime numbers. 
		By normalizing prime gaps using the logarithmic function, representing them in an angular space modulo~30, 
		and discretizing them into a finite alphabet, it becomes possible to construct a Markovian grammar that reveals 
		deep and persistent regularities across multiple numerical scales.
	\end{abstract}
	
	% ============================================================
	% ======================= VERSION FR ==========================
	% ============================================================
	
	\section*{Version française}
	
	\section{Importance mathématique de la grammaire}
	
	La grammaire géométrique des écarts premiers constitue un objet mathématique nouveau. 
	La normalisation logarithmique
	
	
	\[
	g_n = \frac{p_{n+1} - p_n}{\log(p_n)}
	\]
	
	
	révèle une stationnarité remarquable des écarts à travers plusieurs intervalles. 
	Cette stationnarité permet de discrétiser les écarts en un alphabet fini et de construire une grammaire empirique.  
	L'analyse des transitions montre une structure markovienne stable, indiquant que les écarts premiers ne sont pas aléatoires.
	
	\section{Contribution scientifique}
	
	\begin{itemize}
		\item Introduction d'une normalisation géométrique stationnaire.
		\item Construction d'un alphabet symbolique pour les écarts.
		\item Mise en évidence d'une structure markovienne stable.
		\item Découverte d'asymétries modulo~10 et modulo~30.
		\item Développement d'un cadre géométrique unifié.
		\item Création d'un laboratoire logiciel reproductible.
	\end{itemize}
	
	\section{Méthodologie}
	
	\subsection{Extraction des écarts}
	Les écarts sont définis par
	
	
	\[
	A_n = p_{n+1} - p_n.
	\]
	
	
	
	\subsection{Normalisation logarithmique}
	Chaque écart est normalisé par
	
	
	\[
	g_n = \frac{A_n}{\log(p_n)}.
	\]
	
	
	
	\subsection{Discrétisation symbolique}
	Les valeurs normalisées sont classées dans un alphabet discret.
	
	\subsection{Dictionnaires markoviens}
	Deux modèles sont construits : global et conditionné modulo~10.
	
	\subsection{Analyse géométrique modulo 30}
	Les résidus admissibles modulo~30 sont associés à huit directions angulaires.
	
	\subsection{Analyse des motifs}
	Les motifs symboliques de longueur 2, 3 et 4 sont extraits.
	
	\subsection{Cohérence}
	La cohérence entre modèles est mesurée par comparaison des matrices de transition.
	
	\section{Résultats principaux}
	
	\begin{itemize}
		\item Stationnarité des hauteurs normalisées.
		\item Stabilité des proportions symboliques.
		\item Structure markovienne d'ordre~1.
		\item Asymétries arithmétiques persistantes.
		\item Structure géométrique non uniforme.
	\end{itemize}
	
	\section{Conclusion générale}
	
	Cette étude révèle une structure interne stable dans la dynamique des écarts premiers. 
	La combinaison de la normalisation logarithmique, de la géométrie modulo~30, 
	de l'analyse symbolique et des modèles markoviens montre que les écarts premiers 
	suivent une grammaire cohérente et persistante.  
	Cette approche ouvre une nouvelle voie pour l'étude des nombres premiers.
	
	% ============================================================
	% ======================= VERSION EN ==========================
	% ============================================================
	
	\newpage
	\section*{English Version}
	
	\section{Mathematical Importance of the Grammar}
	
	The geometric grammar of prime gaps constitutes a new mathematical object.  
	The logarithmic normalization
	
	
	\[
	g_n = \frac{p_{n+1} - p_n}{\log(p_n)}
	\]
	
	
	reveals a remarkable stationarity of normalized gaps across several intervals.  
	This stationarity enables discretization into a finite alphabet and the construction of an empirical grammar.  
	The transition analysis shows a stable Markov structure, indicating that prime gaps are not random.
	
	\section{Scientific Contributions}
	
	\begin{itemize}
		\item Introduction of a stationary geometric normalization.
		\item Construction of a symbolic alphabet for prime gaps.
		\item Identification of a stable Markov structure.
		\item Discovery of modulo~10 and modulo~30 asymmetries.
		\item Development of a unified geometric framework.
		\item Creation of a reproducible software laboratory.
	\end{itemize}
	
	\section{Methodology}
	
	\subsection{Extraction of gaps}
	Gaps are defined by
	
	
	\[
	A_n = p_{n+1} - p_n.
	\]
	
	
	
	\subsection{Logarithmic normalization}
	Each gap is normalized using
	
	
	\[
	g_n = \frac{A_n}{\log(p_n)}.
	\]
	
	
	
	\subsection{Symbolic discretization}
	Normalized values are classified into a discrete alphabet.
	
	\subsection{Markov dictionaries}
	Two models are constructed: global and modulo~10 conditioned.
	
	\subsection{Geometric analysis modulo 30}
	Admissible residues modulo~30 are mapped to eight angular directions.
	
	\subsection{Pattern analysis}
	Symbolic patterns of length 2, 3, and 4 are extracted.
	
	\subsection{Coherence}
	Coherence between models is measured by comparing transition matrices.
	
	\section{Main Results}
	
	\begin{itemize}
		\item Stationarity of normalized heights.
		\item Stability of symbolic proportions.
		\item First-order Markov structure.
		\item Persistent arithmetic asymmetries.
		\item Non-uniform geometric structure.
	\end{itemize}
	
	\section{General Conclusion}
	
	This study reveals a stable internal structure in the dynamics of prime gaps.  
	The combination of logarithmic normalization, modulo~30 geometry, symbolic analysis, 
	and Markov models shows that prime gaps follow a coherent and persistent grammar.  
	This approach opens a new direction for the study of prime numbers.
	
\end{document}
