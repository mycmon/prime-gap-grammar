\documentclass[12pt]{article}

\usepackage[T1]{fontenc}
\usepackage[utf8]{inputenc}
\usepackage{lmodern}
\usepackage{amsmath, amssymb}
\usepackage{geometry}
\usepackage{setspace}
\usepackage{graphicx}
\usepackage{booktabs}

\geometry{margin=1in}
\setstretch{1.2}

\title{
	\textbf{Une approche empirique des écarts premiers}\
	
	\large Grammaire normalisée, modèle Markovien et observations numériques
}

\author{Michel Monfette  mycmon@gmail.com  (Quebec, Canada)}
\date{Janvier 2026}

\begin{document}
	

	
	\begin{abstract}
		Ce document présente une approche empirique et intuitive pour analyser les écarts entre nombres premiers.
		En normalisant les écarts par $\log(p)$, on observe une structure stable qui peut être discrétisée en une grammaire symbolique.
		Cette grammaire, associée a un modèle Markovien simple, permet de générer des écarts simulés qui reproduisent la texture locale des écarts premiers.
		Les simulations montrent un taux de réussite d'environ 30--41\%, soit environ 4 fois mieux que le hasard dans l'intervalle 1--2 millions.
	\end{abstract}
	
	\section{Introduction}
	
	L’idée initiale provient d'une intuition géométrique simple : les nombres premiers au-delà de 5 n'occupent que 8 positions possibles modulo 30.
	Cette structure a inspire une représentation mentale sous forme de cube $3 \times 3 \times 3$, ou seules certaines positions sont autorisées.
	
	A partir de cette intuition, une observation empirique a été faite : les écarts entre nombres premiers, une fois normalises, semblent suivre des motifs répétitifs et stables.
	
	\section{Normalisation des écarts}
	
	Pour un nombre premier $p_n$, on définit :
	
	
	\[
	\Delta_n = p_{n+1} - p_n,
	\qquad
	g_n = \frac{\Delta_n}{\log(p_n)}.
	\]
	
	
	
	Cette normalisation rend les écarts comparables sur de grands intervalles.
	Les valeurs $g_n$ se regroupent naturellement en classes.
	
	\section{Construction d'une grammaire}
	
	Les valeurs normalisées sont discrétisées en symboles :
	
	\begin{itemize}
		\item a : écart tres serre
		\item b : écart serre
		\item c : écart typique
		\item d : écart large
	\end{itemize}
	
	Les motifs les plus frequents observes sont :
	
	\subsection*{Bigrams}
	bb, ab, ba, bc, cb, aa, ca
	
	\subsection*{Trigrams}
	bbb, bba, bab, abb, bcb, bbc
	
	\subsection*{Tetragrammes}
	bbbb, bbab, bcbb, bbba, cbba, abbb
	
	Ces motifs sont stables sur plusieurs intervalles : 1--2000, 20k--50k, 1M--2M.
	
	\section{Modèle Markovien}
	
	A partir des bigrams, on construit une matrice de transition :
	
	
	\[
	P(s_{n+1} \mid s_n).
	\]
	
	
	
	Cette matrice est stable et permet de générer des séquences symboliques réalistes.
	Chaque symbole est ensuite associe a une valeur typique de $g$, ce qui permet de reconstruire des écarts simulés.
	
	\section{Filtrage arithmétique}
	
	Pour rendre la simulation plus réaliste :
	
	\subsection*{Résidus modulo 30}
	Seuls les résidus :
	
	
	\[
	1,7,11,13,17,19,23,29
	\]
	
	
	sont autorises.
	
	\subsection*{Filtre anti-multiples}
	Les entiers divisibles par :
	
	
	\[
	7, 11, 13, 17, 19, 23, 29
	\]
	
	
	sont élimines.
	
	\section{Résultats expérimentaux}
	
	Sur l'intervalle 1--2 millions, avec 100 valeurs simulées :
	
	\begin{itemize}
		\item 41\% des valeurs simulées sont des nombres premiers
		\item densité réelle : 7.24\%
		\item performance relative : environ $4.1 \times$ le hasard
	\end{itemize}
	
	\section{Comparaison avec le modèle de Granville}
	
	Le modèle de Granville prédit :
	
	
	\[
	P_{\text{Granville}}(n) \approx \frac{1.2}{\ln(n)} \approx 7\%.
	\]
	
	
	
	\begin{center}
		\begin{tabular}{l c}
			\toprule
			Modèle & Résultat \\
			\midrule
			Grammaire normalisée & 41\% \\
			Granville & 7\% \\
			Hasard & 7\% \\
			\bottomrule
		\end{tabular}
	\end{center}
	
	\section{Conclusion}
	
	Cette approche ne constitue pas une théorie, mais une structure empirique stable.
	Elle propose une nouvelle manière de regarder les écarts premiers : non pas comme une suite chaotique, mais comme un langage symbolique avec des motifs répétitifs.
	Ce document vise simplement a partager cette observation avec la communauté mathématique.
	
\end{document}
